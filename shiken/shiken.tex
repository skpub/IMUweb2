\documentclass[uplatex]{jsarticle}
\title{インモラル大学全学部共通入学試験問題}
\author{
    作成: インモラル大学学長 佐藤海音 \\
    試験日時: 今 \\
    諸注意: 1問選択して答えよ。全部答えても良い。
}
\date{\today}

\usepackage{amsmath}
\usepackage{enumitem}
\usepackage{amssymb}
\usepackage{amsfonts}

\begin{document}
\maketitle
\newpage


\section{税制}
100万円, 103万円, 106万円, 130万円どの壁にしても、
そもそも壁を作らずに滑らかに補間しようという発想に至らない役人は
$y=ax$すら活用できないホイ卒である。
\begin{enumerate}
    \item はい
    \item Yes
    \item 是
    \item Ja
\end{enumerate}

\section{音楽}
いわゆる増六の和音の出自として適切なものを答えよ
\begin{enumerate}
    \item 短調のドッペルドミナント
    \item サブスティテュートドミナント(いわゆる裏コード$subV7/V$)
    \item 半音上の調からの借用
    \item アンダルシア進行
\end{enumerate}

\section{線型代数}
線型作用素$T:D(T)\rightarrow R(T)$の単射性と$\mbox{Ker}T=\{0\}$
が同値であることを示せ(ただしこの\{0\}は始域の零ベクトルのみを含む集合を意味する.).

\section{アルゴリズム}
同じ順序集合の元からなる2つのソート済みリスト$A,B$ 及び$A,B$の要素数$N,M$が与えられる。
Aの各元についてBに元として含まれているか検証する疑似コードを記述せよ。含まれている場合にその値を出力すれば良い。
常識的な範囲でフォーマットは問わない。ただし、時間計算量は$O(\mbox{max}(N,M))$で抑えられなくてはならない。


\newpage


\section{現代文}
以下は学長のXでの一連のポスト「能動的シリーズ」である。
\begin{itemize}[label={}]
    \item \begin{tabular}{ll}
        能動的な春の小川&「さらさらいこうかな...」 \\
        能動的な平安人&「係りでも結ぶかぁ~」 \\
        能動的な人材派遣会社&「中でも抜くか〜」 \\
        能動的な夫婦&「暇だな...血でも分かつか...」 \\
        能動的なキス&「...(この人、本当に素敵だな...。唾液を交換したくなってきたぞ...)」 \\
        能動的なエリート&「暇だな…出世でもするか…」 \\
        能動的な一目惚れ&「あっ、永久(とこしえ)の愛でも誓うか...」 \\
    \end{tabular}
\end{itemize}
学長の脳裏にあるこの表現の意図は何か
\begin{enumerate}
    \item 登場人物の気持ちを素直に表現している。
    \item 通常は能動的でない事柄を能動的に考え、やっている人がいたら面白い。
    \item うそはうそであると見抜ける人でないと(Xを使うのは)難しい。
    \item 死は救済である。
\end{enumerate}

\section{化学}
塩化カルシウム水溶液の電気分解を考える。
酸素ではなく塩素を多く発生させるための工夫として適切なものを選べ(複数回答)
\begin{enumerate}
    \item 陰極にステンレスを使う
    \item 過電圧を考えて陽極に白金電極ではなく黒鉛電極を使う
    \item 電圧を高くする
    \item 電解液に3年お湯を交換していないマコモ湯を加える
    \item 水酸化物イオンの酸化反応で酸素が出る反応は起こりやすいため、これを避けるために液性を酸性にする
\end{enumerate}

\section{物理}
地面水平方向に対して角度が$\theta$の坂の上に物体があって
力を坂の進行方向と地面向きに$mg$サインコサイン$\theta$のやつの図を描け。

\section*{試験回答方法・提出方法}

紙に書いて撮るでも良いし、\LaTeX で書いてpdfに変換しても良い。
問題によっては紙に書くまでもないので、
その場合は後述するようにDMやリプライでそのまま送りつければ良い。

ただし、Microsoft Wordの.docxは絶対に許さない(pdfに変換すれば良し)。

提出に関しては、何らかの方法で学長の手に渡れば良い。

Xの\texttt{ @OMGR\_dearinsu }にDMやリプライで送ることを推奨する。

\end{document}